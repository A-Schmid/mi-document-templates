\documentclass{mi-graduation}

\bachelor % im Falle einer Masterarbeit \master

% Variablen, die für das Deckblatt und Metadaten verwendet werden
\title{[Titel der Bachelor-/ Masterarbeit]}
\author{[Autor*in der Arbeit]}
\semester{[WS / SS und Jahreszahl]}
\course{[Art des Seminars und Seminartitel (z.B. Praktikum Multimedia Engineering)]}
\module{[z.B. MEI-M 04 (B.A.)]}
\dozent{[Seminarleiter]}
\studid{[Matrikelnummer]}
\studSemester{[Semesterzahl und Studiengänge (z.B. 3. Semester B.A. Medieninformatik / Informationswissenschaft)]}
\phone{0941/133742666} % Optional
\studSubject{Medieninformatik}
\firstReviewer{Prof. Dr. Maike Musterprof}
\secondReviewer{Prof. Dr. Max Musterprof}
\advisor{Momo Mustermensch}
\address{Domplatz 1, 93047 Regensburg}{} % Optional
\mail{[Emailadresse (z.B.: max.mustermann@stud.uni-regensburg.de)]}
\studMail{[Emailadresse (z.B.: max.mustermann@stud.uni-regensburg.de)]}
\dateHandedIn{[Abgabetermin der Arbeit]}
\keywords{Enter;key;words;here}
\writemeta

\begin{document}

% Das Deckblatt erstellen
\maketitle

% Die Nummerierung beginnt mit der Titelseite (= Seite 1), soll aber erst ab der ersten Inhaltsseite (Einleitung) angezeigt werden.
\pagestyle{empty}

%\defaultStretch % Zeilenabstand auf 1.5
\doublespacing % doppelter Zeilenabstand sieht mehr nach 1.5 fachem aus
\newpage

\tableofcontents % Optional
\newpage
\listoffigures % Optional
\newpage
\lstlistoflistings % Optional
\newpage

\summary
Bachelor- und Masterarbeiten beginnen mit einer Zusammenfassung in einer deutschen und englischen Version.
Die Zusammenfassung gibt einen Überblick über Thema und Resultate der Arbeit.
Inhaltlich werden die Zielsetzung, die Methodik, die einzelnen Arbeitsschritte bzw. Gliederungspunkte und die Ergebnisse der Arbeit widergegeben.

Schlecht: „Schon immer haben Menschen Zusammenfassungen geschrieben [Platitüde, keine Zusammenfassung]. In dieser Arbeit wurde in mehreren Studien untersucht, wie Zusammenfassungen wirken. [Was genau wurde untersucht? Was waren die Ergebnisse?]“

Besser: „In dieser Arbeit wurde untersucht, inwiefern das Lesen von Zusammenfassungen das Lesen des kompletten Dokuments ersetzen kann. Dazu wurden zwei Studien mit jeweils 17 Teilnehmern durchgeführt. In der ersten wurde [...]. Diese Ergebnisse zeigen, dass Bedienungsanleitungen und Bilderbücher weniger gut über Zusammenfassungen erschlossen werden können, als Romane oder Sachbücher.“

\abstract
A summary in English. It should be more or less similar to the German Zusammenfassung. Avoid too verbatim translations („In this work it was examined how the reading of ...“)

\newpage
\pagestyle{fancy}

\section{Über dieses Dokument}
Dieses Dokument soll Ihnen den Einstieg beim Verfassen einer Studienarbeit erleichtern.
Die Vorgaben sind als Empfehlungen zu verstehen, können aber bei Bedarf in Absprache mit den Dozenten angepasst und erweitert werden.
Hier steht ein Beispielabschnitt.
Die Schriftart ist Palatino Linotype, die Schriftgröße 11pt.
Ein Verweis auf eine Abbildung (vgl. \nameref{img:norman2010}) wird in diesem Satz verdeutlicht.
Alternativ eignen sich auch Serifenschriftarten wie Garamond, Times New Roman oder Frutiger Serif Pro.
Querverweise (vgl. \nameref{section:2}) werden in diesem Satz  gezeigt.
Der erste Absatz jedes Abschnitts und Absätze nach Abbildungen werden nicht eingerückt (Formatvorlage ist Standard).
Ein weiterer Absatz wird durch Drücken der Return-Taste erzeugt und automatisch eingerückt.
Absätze werden also nicht durch Leerzeilen, sondern durch Einrücken des Folgeabsatzes getrennt.
Die Folgeabsätze erhalten automatisch die Formatvorlage Folgeabsatz.
In dieser Vorlage können auch Codeschnipsel eingesetzt und referenziert werden (vgl. Codebeispiel: \nameref{helloworld}).

In der Kopfzeile erscheint immer der Text der aktuellen Überschrift, die mit der Formatvorlage „Überschrift 1“ formatiert wird. Somit wird dem Leser die Orientierung in der Arbeit erleichtert.\\

\begin{figure}[h!]
    \center{
	    \img{demo.jpg}{300pt}
	}
	\caption{Blümchen \citep{Norman:2002}}
	\label{img:norman2010}
\end{figure}

Anschließend noch zwei Unterüberschriften und ein Codebeispiel.

\subsection{Abschnitt 2}\label{section:2}
Ein Beispiel für ein Codeschnipsel in Python.\\

\begin{lstlisting}[captionpos=b, belowcaptionskip=4pt, caption=Hello World (Python), label=helloworld, language=Python]
print("Hello World")
\end{lstlisting}

Ein Beispiel für ein Codeschnipsel in Java.\\

\begin{lstlisting}[captionpos=b, belowcaptionskip=4pt, caption=Hello World (Java), language=Java]
public class Main {
	public static void main(String[] args){
		System.out.println("Hello World");
	}
}
\end{lstlisting}

\newpage
% Literaturverzeichnis anzeigen
\bibliographystyle{apacite}
\bibliography{literature}

\newpage
% Erklärung zur Urherberschaft (urheberschaft.tex) anhängen
\addcontentsline{toc}{section}{Erklärung zur Urheberschaft}
\section*{Erklärung zur Urheberschaft}

\noindent
Ich habe die Arbeit selbständig verfasst, keine anderen als die angegebenen Quellen und Hilfsmittel benutzt, sowie alle Zitate und Übernahmen von fremden Aussagen kenntlich gemacht. 

\noindent
Die Arbeit wurde bisher keiner anderen Prüfungsbehörde vorgelegt. 

\noindent
Die vorgelegten Druckexemplare und die vorgelegte digitale Version sind identisch.

\noindent
Von den zu § 27 Abs. 5 der Prüfungsordnung vorgesehenen Rechtsfolgen habe ich Kenntnis.

\signature


\newpage
% Erklärung zur Lizenzierung der Arbeit (lizenzierung.tex) anhängen
\addchap{Erklärung zur Lizenzierung und Publikation dieser Arbeit}

\textbf{Name:} \getAuthor

\textbf{Titel der Arbeit:} \textit{\getTitle}

Hiermit gestatte ich die Verwendung der schriftlichen Ausarbeitung zeitlich unbegrenzt und nicht-exklusiv unter folgenden Bedingungen:

\begin{itemize}
    \item[\checkboxEmpty] Nur zur Bewertung dieser Arbeit
    \item[\checkboxEmpty] Nur innerhalb des Lehrstuhls im Rahmen von Forschung und Lehre
    \item[\checkboxChecked] Unter einer Creative-Commons-Lizenz mit den folgenden Einschränkungen:
    \begin{itemize}
        \item[\checkboxChecked] BY – Namensnennung des Autors
        \item[\checkboxEmpty] NC – Nichtkommerziell
        \item[\checkboxEmpty] SA – Share-Alike, d.h. alle Änderungen müssen unter die gleiche Lizenz gestellt werden.
    \end{itemize}
\end{itemize}
{\scriptsize(An Zitaten und Abbildungen aus fremden Quellen werden keine weiteren Rechte eingeräumt.)}

Außerdem gestatte ich die Verwendung des im Rahmen dieser Arbeit erstellten Quellcodes unter folgender Lizenz:

\begin{itemize}
    \item[\checkboxEmpty] Nur zur Bewertung dieser Arbeit
    \item[\checkboxEmpty] Nur innerhalb des Lehrstuhls im Rahmen von Forschung und Lehre
    \item[\checkboxEmpty] Unter der CC-0-Lizenz (= beliebige Nutzung)
    \item[\checkboxChecked] Unter der MIT-Lizenz (= Namensnennung)
    \item[\checkboxEmpty] Unter der GPLv3-Lizenz (oder neuere Versionen)
\end{itemize}

{\scriptsize(An explizit mit einer anderen Lizenz gekennzeichneten Bibliotheken und Daten werden keine weiteren Rechte eingeräumt.)}

\noindent
Ich willige ein, dass der Lehrstuhl  für Medieninformatik diese Arbeit – falls sie besonders gut ausfällt - auf dem Publikationsserver der Universität Regensburg veröffentlichen lässt.

\noindent
Ich übertrage deshalb der Universität Regensburg das Recht, die Arbeit elektronisch zu speichern und in Datennetzen öffentlich zugänglich zu machen. Ich übertrage der Universität Regensburg ferner das Recht zur Konvertierung zum Zwecke der Langzeitarchivierung unter Beachtung der Bewahrung des Inhalts (die Originalarchivierung bleibt erhalten).

\noindent
Ich erkläre außerdem, dass von mir die urheber- und lizenzrechtliche Seite (Copyright) geklärt wurde und Rechte Dritter der Publikation nicht entgegenstehen.

\begin{itemize}
    \item[\checkboxChecked] Ja, für die komplette Arbeit inklusive Anhang
    \item[\checkboxEmpty] Ja, für eine um vertrauliche Informationen gekürzte Variante (auf dem Datenträger beigefügt)
    \item[\checkboxEmpty] Nein
    \item[\checkboxEmpty] Sperrvermerk bis (Datum):
    %Sperrvermerke sind mit dem Betreuer am Lehrstuhl abzustimmen. Sperrvermerke mit einer Frist von mehr als zwei Jahren benötigen immer eine schriftliche Begründung, aus der hervorgeht, weshalb eine kürzere Sperrfrist nicht ausreichend ist.
\end{itemize}

\signature


\end{document}
